% Created 2016-01-19 Tue 09:21
\documentclass[11pt]{article}
\usepackage[utf8]{inputenc}
\usepackage[T1]{fontenc}
\usepackage{fixltx2e}
\usepackage{graphicx}
\usepackage{grffile}
\usepackage{longtable}
\usepackage{wrapfig}
\usepackage{rotating}
\usepackage[normalem]{ulem}
\usepackage{amsmath}
\usepackage{textcomp}
\usepackage{amssymb}
\usepackage{capt-of}
\usepackage{hyperref}
\usepackage{tikz}
\usepackage{fancyhdr}
\usepackage[left=2cm,right=2cm,top=2cm,bottom=2cm]{geometry}
\newcommand\leftidx[3]{{\vphantom{#2}}#1#2#3}
\pagestyle{fancyplain}
\chead{{\it Nuclear Reactor Physics}}
\cfoot{{\it ENGY 5050, UMass Lowell}}
\author{Justin Pounders}
\date{\today}
\title{Nuclear Physics in 60 Seconds}
\hypersetup{
 pdfauthor={Justin Pounders},
 pdftitle={Nuclear Physics in 60 Seconds},
 pdfkeywords={},
 pdfsubject={},
 pdfcreator={Emacs 24.5.1 (Org mode 8.3.2)}, 
 pdflang={English}}
\begin{document}

\maketitle
\tableofcontents

\textbf{Notation:} An atomic nucleus is the small, dense core of an atom, consisting of a collection of protons and neutrons.  The number of protons contained within a given nucleus is given by the atomic number, \(Z\), while the number of neutrons is given by \(N\).  The mass number, \(A\), is the sum of the number of neutrons and protons.  A neutral atom, \(X\), is typically written with the \(A\) and \(Z\) numbers prepended, i.e., \(\leftidx{^A_Z}{X}{}\), with the neutron number implicit.

\textbf{Mass:} Masses on the nuclear scale at typically expressed in units of \emph{atomic mass units} (u), defined so that the mass of a neutron atom of \(\leftidx{^12_6}{C}{}\) is exactly 12 u.

\textbf{Quantum Description:}  A common and quite accurate quantum description of the nucleus is given by the \emph{shell model}, which is analogous to the description of atomic electrons.  Neutrons, protons and electrons are all classified as \emph{fermions}, which are particles with a spin of \(\frac{1}{2}\hbar\) that obey the Pauli exclusion principle.  Neutrons and protons in a nucleus reside in discrete energy states and posses angular momentum that also occurs in discrete amounts.  The angular momentum is specified by the positive, integer quantum nunber \(\ell \geq 0\), and the first few angular momentum states are are labeled \(s\), \(p\), \(d\), \(f\), etc, again in analogy to atomic electrons.  The \emph{ground state} of a nucleus occurs when all of the nuclear particles are in the lowest energy states allowed by the Pauli exclusion principle.  An \emph{excited state} occurs when a nucleon is elevated to a higher (and unstable) energy level.
\end{document}