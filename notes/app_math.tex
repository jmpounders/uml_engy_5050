% Created 2016-02-15 Mon 13:28
\documentclass[11pt]{article}
\usepackage[utf8]{inputenc}
\usepackage[T1]{fontenc}
\usepackage{fixltx2e}
\usepackage{graphicx}
\usepackage{grffile}
\usepackage{longtable}
\usepackage{wrapfig}
\usepackage{rotating}
\usepackage[normalem]{ulem}
\usepackage{amsmath}
\usepackage{textcomp}
\usepackage{amssymb}
\usepackage{capt-of}
\usepackage{hyperref}
\usepackage{tikz}
\usepackage{fancyhdr}
\usepackage[left=2cm,right=2cm,top=2cm,bottom=2cm]{geometry}
\renewcommand\vec{\mathbf}
\newcommand\leftidx[3]{{\vphantom{#2}}#1#2#3}
\newenvironment{example}[1]{\vspace{0.2in}\hrule\vspace{0.1in}\noindent\emph{Example:} #1 \\}{\vspace{0.1in}\hrule\vspace{0.2in}}
\pagestyle{fancyplain}
\cfoot{{\it ENGY 5050, Nuclear Reactor Physics, UMass Lowell}}
\author{Justin Pounders}
\date{\today}
\title{Mathematical Odds and Ends}
\hypersetup{
 pdfauthor={Justin Pounders},
 pdftitle={Mathematical Odds and Ends},
 pdfkeywords={},
 pdfsubject={},
 pdfcreator={Emacs 24.5.1 (Org mode 8.3.2)}, 
 pdflang={English}}
\begin{document}

\maketitle
\tableofcontents

\section{Trigonometric Identities and Relationships}
\label{sec:orgheadline4}
\subsection{Euler's Formula}
\label{sec:orgheadline1}
Euler's formula is given by
\begin{align}
  e^{i x} = \cos x + i \sin x
\end{align}
where \(e\) is the base of the natural logarithm and \(i = \sqrt{-1}\).  Many of the standard trigonometric identities can be quickly derived from this one formula.
\subsection{Additive Identities}
\label{sec:orgheadline2}
\begin{align}
  \sin\left(A \pm B\right) = \sin A \cos B \pm \cos A \sin B
\end{align}
\begin{align}
  \cos\left(A \pm B\right) = \cos A \cos B \mp \sin A \sin B
\end{align}
\subsection{Law of Sines and Cosines}
\label{sec:orgheadline3}
Given the triangle shown in Figure \ref{fig::triangle}, the \emph{law of sines} states that
\begin{align}
  \frac{a}{\sin A} = \frac{b}{\sin B} = \frac{c}{\sin C} = \text{ constant} \;\;.
\end{align}
The \emph{law of cosines} states that
\begin{align}
  c^2 = a^2 + b^2 - 2ab\cos{C}
\end{align}

\begin{figure}
\centering
\begin{tikzpicture}[x=0.25in,y=0.25in,scale=0.5]
  \draw (0,0) -- (23,0) -- (17,15) -- (0,0);
  \draw (11.5,0) node [anchor=north] {$a$};
  \draw (20,7.5) node [anchor=south west] {$b$};
  \draw (8.5,7.5) node [anchor=south east] {$c$};

  \draw (17.5,13.5) arc [start angle=-68.1986, end angle=-138.5673, radius=1.5];
  \draw (16.5,12.5) node {$A$};

  \draw (1.5,0) arc [start angle=0, end angle=41.424, radius=1.5];
  \draw (2.5,1) node {$B$};

  \draw (21.5,0) arc [start angle=180, end angle=111.8014, radius=1.5];
  \draw (20.5,1) node {$C$};
\end{tikzpicture}
\caption{A triangle.}
\label{fig::triangle}
\end{figure}
\section{Special Functions}
\label{sec:orgheadline7}
\subsection{Delta Functions}
\label{sec:orgheadline5}
The Dirac delta function is a generalized function defined by
\begin{align}
  \int_{-\infty}^\infty \delta(x-a) f(x) dx = f(a)
\end{align}
for some function \(f(x)\), and
\begin{align}
  \delta(x-a) = 
  \begin{cases}
    0, & x\neq a, \\
    \text{undefined}, & x=a \;\;.
  \end{cases}
\end{align}
The fact that the Delta function is a \emph{generalized} function means that it is only really defined with respect to integration.  Pragmatically this presents no difficulty if it used in probability density functions or other functions that must be integrated to yield physically meaningful results.

The discrete analog of the Dirac delta function is the Kronecker delta, defined
\begin{align}
  \delta_{i,j} = 
  \begin{cases}
    0 & i \neq j, \\
    1 & i = j.
  \end{cases}
\end{align}
\subsection{Legendre Polynomials}
\label{sec:orgheadline6}
The Legendre polynomials are a set of polynomials, \(\left\{ P_n \right\}_{n=0}^\infty\).  For a given \(n\), the polynomial \(P_n(x)\) is a polynomial of degree \(n\).  For example, the first few Legendre polynomials are
\begin{align*}
  P_0(x) &= 1, \\
  P_1(x) &= x, \\
  P_2(x) &= \frac{1}{2}\left(3 x^2 - 1 \right), \\
  P_3(x) &= \frac{1}{2}\left(5 x^3 - 3 x \right), \\
  P_4(x) &= \frac{1}{8}\left(34 x^4 - 30 x^2 + 3 \right).
\end{align*}

The Legendre polynomials are orthogonal over the interval \((-1, 1)\) and satisfy the following orthogonality relationship:
\begin{align}
  \int_{-1}^1 P_n(x) P_m(x) dx = \frac{2}{2n+1} \delta_{n,m},
\end{align}
where \(\delta_{n,m}\) is the Kronecker delta.

The polynomials also satisfy the following two recurrence relations:
\begin{align}
  (n+1) P_{n+1}(x) - (2n + 1) x P_n(x) + n P_{n-1}(x) = 0, \\
  \left(1 - x^2\right) \frac{dP_n}{dx} = -n x P_n(x) + n P_{n-1}(x).
\end{align}

The Legendre polynomials form a complete set over the interval \([-1, 1]\).  This means that a bounded, piecewise-continuous function \(f(x)\) may be written as linear combination of the Legendre polynomials.  Namely,
\begin{align}
  f(x) = \sum_{n=0}^\infty f_n P_n(x), \quad x \in [-1,1]
\end{align}
where
\begin{align}
  f_n = \frac{2n+1}{2} \int_{-1}^{1} f(x) P_n(x) dx.
\end{align}
\section{The Direction Vector}
\label{sec:orgheadline8}
The unit direction vector can be expressed in terms of a polar angle, \(\theta\), and a azimuthal angle, \(\varphi\), as shown in Figure \ref{fig::unitDirection}.  In a Cartesian coordinates the unit direction vector can be written
\begin{align}
  \vec{\Omega} = \Omega_x \vec{i} + \Omega_y \vec{j} + \Omega_z \vec{k}
\end{align} 
where
\begin{subequations}
\begin{align}
  \Omega_x = \sin\theta \cos\varphi, \\
  \Omega_y = \sin\theta \sin\varphi, \\
  \Omega_z = \cos\theta.
\end{align}
\end{subequations}
The differential solid angle can be written
\begin{align}
  d\vec{\Omega} = \sin\theta d\varphi d\theta.
\end{align}
The direction vector is said to subtend \(4\pi\) steradians, as
\begin{align}
  \int_0^\pi d\theta \sin\theta \int_0^{2\pi} d\varphi = 4\pi.
\end{align}
For this reason, integration over all solid angles is commonly denoted \(\int_{4\pi} d\vec{\Omega}\).
It is common to make the change of variable \(\theta \rightarrow \mu = \cos\theta\).  In this case
\begin{align}
  \int_{4\pi} d\vec{\Omega} = \int_{-1}^1 d\mu \int_0^{2\pi} d\varphi.
\end{align}

\begin{figure}
\centering
\begin{tikzpicture}[x=0.25in,y=0.25in,scale=0.8]
  \draw [->] (0,0) -- (12,0) node [anchor=south east] {$x$};
  \draw [->] (0,0) -- (-5,-6) node [anchor=south east] {$y$};
  \draw [->] (0,0) -- (0,8) node [anchor=north east] {$z$};

  \draw [->,very thick] (0,0) -- (7,4) node [anchor=south east] {\Large $\vec{\Omega}$};

  \draw [dashed] (0,0) -- (7,-5);
  \draw [dashed] (7,4) -- (7,-5);
  \draw [dashed] (7,-5) -- (7+25/6,0);
  \draw [dashed] (7,4) -- (0,4+5/7*4);

  \draw (1,4/7) arc [start angle=29.74, end angle=90, radius=1.15];
  \draw (0.6,1.3) node [anchor=south west] {\large $\theta$};

  \draw (1,-5/7) arc [start angle=324.46,end angle=360, radius=1.23];
  \draw (1.65,-0.65) node [anchor=west] {\large $\varphi$};
\end{tikzpicture}
\caption{Description of the unit direction vector.}
\label{fig::unitDirection}
\end{figure}

It is common for directions to be described probabilistically, given some probability density \(p\left(\vec{\Omega}\right)\).  For example, if a direction is selected isotropically, then we have
\begin{align}
  p\left(\vec{\Omega}\right) = \frac{1}{4\pi}.
\end{align}
In one-dimensional problems the isotropic distribution can be reduced to a function of \(\theta\) alone.  This is done by integrating over the circles of radius \(\sin\theta\) formed by tracing out \(\varphi\) over \(2\pi\) angles for each \(\theta\):
\begin{align}
  p(\theta) = \int_0^{2\pi} p\left(\vec{\Omega}\right) \sin\theta d\varphi
            = \frac{1}{2} \sin\theta.
\end{align}
This distribution, for example, describes isotropic neutron emission in a CM scattering event.

\section{Numerical Integration Quadrature}
\label{sec:orgheadline11}
\subsection{Midpoint Rule}
\label{sec:orgheadline9}
Perhaps the simplest quadrature is based on the midpoint rule.  This is essentially the method of "rectangles."  Given an interval \([a,b]\) and a set of points \(x_i\) for \(i=0,\hdots,n\) with \(x_{i-1} < x_i\), \(x_0 = a\) and \(x_n = b\) we have the following approximate evaluation of an integral:
\begin{align}
  \int_a^b f(x) \approx \sum_{i=1}^n \left( x_i - x_{i-1} \right) f\left(\frac{x_i+x_{i-1}}{2}\right).
\end{align}
If the spacing is uniform with \(h = x_i - x_{i-1}\) then we have
\begin{align}
  \int_a^b f(x) \approx \sum_{i=1}^n h f\left(a + ih/2\right).
\end{align}
\subsection{Gauss-Hermite Quadrature}
\label{sec:orgheadline10}
The Gauss-Hermite quadrature is a form of Gaussian quadrature for approximating integrals of the form
\begin{align}
  \int_{-\infty}^\infty e^{-x^2} f(x) dx.
\end{align}
The approximate evaluation of this integral using the quadrature is
\begin{align}
  \int_{-\infty}^\infty e^{-x^2} f(x) dx \approx \sum_{i=1}^n w_i f(x_i)
\end{align}
where the \(x_i\) are the roots of the degree-\(n\) Hermite Polynomial, \(H_n(x)\), and the \(w_i\) are the associated weights given by
\begin{align}
  w_i = \frac{2^{n-1} n! \sqrt{\pi}}{n^2 H_{n-1}^2(x_i)}.
\end{align}

In Python the appropriate quadrature may be generate by
\begin{verbatim}
import numpy as np
(x,w) = np.polynomial.hermite.hermgauss(n)
\end{verbatim}
where \texttt{n} is the degree of the polynomial, \texttt{x} is a numpy array of the ordinates, and \texttt{w} is a numpy array of the weights.

Matlab does not have a built in a function for computing the quadrature.  The following function was extracted from a function uploaded by Geert Van Damme to \href{http://www.mathworks.com/matlabcentral/fileexchange/26737-legendre-laguerre-and-hermite-gauss-quadrature/content/GaussHermite.m}{Matlab Central File Exchange} under a BSD License.
\begin{verbatim}
function [x, w] = GaussHermite(n)
% This function determines the abscisas (x) and weights (w) for the
% Gauss-Hermite quadrature of order n>1, on the interval [-INF, +INF].
    % This function is valid for any degree n>=2, as the companion matrix
    % (of the n'th degree Hermite polynomial) is constructed as a
    % symmetrical matrix, guaranteeing that all the eigenvalues (roots)
    % will be real.

%  Geert Van Damme
% geert@vandamme-iliano.be
% February 21, 2010    

% Building the companion matrix CM
    % CM is such that det(xI-CM)=L_n(x), with L_n the Hermite polynomial
    % under consideration. Moreover, CM will be constructed in such a way
    % that it is symmetrical.
i   = 1:n-1;
a   = sqrt(i/2);
CM  = diag(a,1) + diag(a,-1);

% Determining the abscissas (x) and weights (w)
    % - since det(xI-CM)=L_n(x), the abscissas are the roots of the
    %   characteristic polynomial, i.d. the eigenvalues of CM;
    % - the weights can be derived from the corresponding eigenvectors.
[V L]   = eig(CM);
[x ind] = sort(diag(L));
V       = V(:,ind)';
w       = sqrt(pi) * V(:,1).^2;
\end{verbatim}
\end{document}