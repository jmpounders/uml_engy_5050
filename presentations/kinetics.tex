% Created 2016-04-25 Mon 12:39
\documentclass[presentation]{beamer}
\usepackage[utf8]{inputenc}
\usepackage[T1]{fontenc}
\usepackage{fixltx2e}
\usepackage{graphicx}
\usepackage{grffile}
\usepackage{longtable}
\usepackage{wrapfig}
\usepackage{rotating}
\usepackage[normalem]{ulem}
\usepackage{amsmath}
\usepackage{textcomp}
\usepackage{amssymb}
\usepackage{capt-of}
\usepackage{hyperref}
\usetheme{Warsaw}
\author{Justin Pounders}
\date{\today}
\title{Reactor Kinetics}
\hypersetup{
 pdfauthor={Justin Pounders},
 pdftitle={Reactor Kinetics},
 pdfkeywords={},
 pdfsubject={},
 pdfcreator={Emacs 24.5.1 (Org mode 8.3.2)}, 
 pdflang={English}}
\begin{document}

\maketitle

\begin{frame}[label={sec:orgheadline1}]{Point Kinetics}
\begin{block}{PKE}
\begin{align*}
  \frac{dP}{dt} &= \frac{\rho - \bar{\beta}}{\Lambda}P + \sum_{i=1}^6 \lambda_i C_i \\
  \frac{dC_i}{dt} &= \frac{\bar{\beta}_i}{\Lambda} P - \lambda_i C_i
\end{align*}
\end{block}
\end{frame}
\begin{frame}[label={sec:orgheadline2}]{Inhour Equation}
Assume constant reactivity and a solution of the form
\begin{align*}
  P(t) &= P_0 e^{\omega t} \\
  C_i(t) &= C_{i,0} e^{\omega t}
\end{align*}
\pause
Plugging into the PKE leads to
\begin{align*}
  \omega P_0 e^{\omega t}  &= \frac{\rho - \bar{\beta}}{\Lambda}P_0 e^{\omega t}  + \sum_{i=1}^6 \lambda_i C_{i,0} e^{\omega t} \\
  \omega C_{i,0} e^{\omega t} &= \frac{\bar{\beta}_i}{\Lambda} P_0 e^{\omega t}  - \lambda_i C_{i,0} e^{\omega t}
\end{align*}
\pause
Combining the two equations results in the \emph{inhour} equation
\begin{align*}
  \rho = \Lambda \omega + \sum_{i=1}^6 \frac{\bar{\beta}_i \omega}{\omega + \lambda_i}
\end{align*}
\end{frame}
\begin{frame}[label={sec:orgheadline3}]{Roots of the Inhour Equation}
There are 7 roots of the inhour equation (they are the eigenvalues of the PKE \emph{system matrix}:
\begin{align*}
  \rho = \Lambda \omega_j + \sum_{i=1}^6 \frac{\bar{\beta}_i \omega_j}{\omega_j + \lambda_i}, \;\;\; j = 1,2,\hdots, 7
\end{align*}
\pause
This means we have a solution of the form
\begin{align*}
  P(t) = \sum_{j=0}^6 a_j e^{\omega_j t}
\end{align*}
Six of the roots will always be negative
\begin{align*}
  \omega_j < -\lambda_j, \quad j=1,2,\hdots, 6
\end{align*}
The root \(\omega_{\text{0}}\) may be positive or negative.
\begin{align*}
  P(t) = a_0 e^{\omega_0 t} + \sum_{j=1}^6 a_j e^{-|\omega_j| t}
\end{align*}
\end{frame}
\begin{frame}[label={sec:orgheadline4}]{Asymptotic Dynamics}
Asymptotically (\(t >> 0\)) we have
\begin{align*}
  P(t) \approx a_0 e^{\omega_0 t} \sim e^{t/T}
\end{align*}
where the (asymptotic) period is defined as
\begin{align*}
  T = \omega_0^{-1}
\end{align*}
\end{frame}
\begin{frame}[label={sec:orgheadline5}]{Subcritical Limit}
As \(\rho \rightarrow -\infty\)
\begin{align*}
  T \rightarrow \lambda_1^{-1} \approx 80 \text{ seconds (U-235)}
\end{align*}
\end{frame}
\begin{frame}[label={sec:orgheadline6}]{Supercritical (I)}
For \(0 \leq \rho \leq \bar{\beta}/2\),
\begin{align*}
  \omega_0 + \lambda_i \approx \lambda_i
\end{align*}
\pause
so
\begin{align*}
  \rho &=  \omega_0 \left[\Lambda + \sum_{i=1}^6 \frac{\bar{\beta}_i}{\lambda_i} \right] \\
       &\approx \omega_0 \times 0.1 \text{ seconds}
\end{align*}
\pause
This means that
\begin{align*}
  T = 0.1 \rho^{-1} \text{ seconds} \approx \frac{1}{4} \frac{\bar{\beta}}{\rho} \text{ minutes}
\end{align*}
\end{frame}
\begin{frame}[label={sec:orgheadline7}]{Supercritical (II)}
For \(\rho > \bar{\beta}\),
\begin{align*}
  \omega_0 + \lambda_i \approx \omega_0
\end{align*}
\pause
so
\begin{align*}
  \rho = \Lambda \omega_0 + \bar{\beta}
\end{align*}
\pause
This means that
\begin{align*}
  T = \frac{\Lambda}{\rho-\bar{\beta}} \approx 10^{-2} \left(\frac{\rho}{\bar{\beta}}-1\right)^{-1}
\end{align*}
\end{frame}
\end{document}